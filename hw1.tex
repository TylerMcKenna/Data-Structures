\documentclass{homework}
\author{Tashfeen, Ahmad}
\class{CSCI 2114: Data Structures}
% \date{\today}
\title{Homework 1}
\address{Oklahoma City University, Petree College of Arts \& Sciences, Computer Science}

\usepackage{fontspec}
\setmonofont[Scale=MatchLowercase]{Iosevka SS07}

\begin{document} \maketitle

\question Write a Java program that prints out one line of text to the
console. It can be anything but ``Hello World!''

\question Write a Java program that populates an array of size $n$
with the first $n$ Fibonacci numbers. The program should print out the
array as shown in figure \ref{exmp}. Here $n$ should be the first command line
argument.

\question Using the
\href{https://en.wikipedia.org/wiki/Sieve_of_Eratosthenes}{Sieve of
  Eratosthenes}, populate a boolean array of size $n$ marking all the
indices that are Prime numbers. Here $n$ should be the first command
line argument.

\begin{enumerate}
  \item For debugging, have your program print all the prime numbers
        less than a 100. You should get the following,
        \[
          2, 3, 5, 7, 11, 13, 17, 19, 23, 29, 31, 37, 41, 43, 47,
          53, 59, 61, 67, 71, 73, 79, 83, 89, 97
        \]
  \item The program should print out at most the five largest
        prime numbers it computed and the time (seconds) it took to
        compute all the primes less than $n$. Here is a way to compute
        seconds taken by a function call
        \texttt{eratosthenes(toSieve)}.

        \begin{lstlisting}[language=java]
double startTime = System.nanoTime();
eratosthenes(toSieve);
double duration = System.nanoTime() - startTime;
duration = duration / Math.pow(10, 9);
\end{lstlisting}

  \item With your program, calculate how long does it take (in seconds)
        to compute all the 30 bit prime numbers. These are all primes less
        than $n = 2^{30} = 1073741824$.

  \item Can your implementation of the Sieve of Eratosthenes
        compute all the 32 bit prime numbers? If yes, give the time it
        takes or if it can not, then why not?
\end{enumerate}

\question Read all bytes in the file \href{https://tinyurl.com/24bvsnaf}{\texttt{half\_gaps.bin}}. You may
use the function in code listing \ref{byte}.

\lstinputlisting[
  linerange={31-39},
  language={java},
  caption={Java function to read in a file's bytes (as signed).},
  label=byte]
{code/FourTashfeen.java}

The function in code listing \ref{byte} reads in signed bytes. While
this maybe suitable for some binary arrangements, we want the bytes to
be unsigned. One way to achieve this is to just loop and use
\href{https://tinyurl.com/2aw28b6l}{\texttt{Byte.toUnsignedLong(byte x)}}
as seen in listing \ref{long}.

\lstinputlisting[
  linerange={9-12},
  language={java},
  caption={Converting Java signed bytes to unsigned longs.},
  label=long]
{code/FourTashfeen.java}

Note that while we may also use
\href{https://tinyurl.com/25w2e4wy}{\texttt{Byte.toUnsignedInt(byte x)}}
to turn the signed bytes into unsigned ints instead of longs, we
prefer longs here to avoid overflows in future computations.

Compute the array of longs' cumulative sum, i. e.,
\[
  \curl{x_i \in \mathtt{cumsum}(x) : x_i = \sum_{k=1}^{i} x_k}
\]
Now multiply each of the sums with 2 and then add a 3.
\[
  \curl{x_i \in \mathtt{cumsum}(x) : y_i = 2x_i+3 = 2\paren{\sum_{k=1}^{i} x_k}+3}
\]
\begin{enumerate}
  \item Print out the first fifteen and the last five elements of this final
        array.
  \item Time this program (the reading of bytes, the cumulative sum
        computation and the doubling with adding a three) and print the
        result in seconds.
  \item Do you recognise the printed numbers? What will these be
        if we further added a 2 and a 3 to them?
\end{enumerate}

\question Break the Affine cipher. Your professor encrypted a plain
text file called \texttt{plain.txt} using the program given in listing
\ref{cipher}.

\lstinputlisting[
  language={java},
  caption={An Affine cipher in Java.},
  label=cipher]
{code/FiveTashfeen.java}

He then redirected the output to a cipher file called
\href{https://tinyurl.com/24pjud2t}{\texttt{cipher.txt}}.

\begin{enumerate}
  \item Use the cipher text file and the code in listing
        \ref{cipher} to recover the plain text. \textit{Hint}: $7^{-1} = 55 \mod 2^7$.
  \item What should the $2^7$ tell you about the text encoding of
        the original plain text file?
\end{enumerate}

\section{Example Executions}

Figure \ref{exmp} shows how the output of the code for the first three
questions should look like on the standard out. All your programs must
compile/run from the command line using \texttt{javac} and
\texttt{java} commands, e. g.,

\begin{verbatim}
javac Program.java
java Program
\end{verbatim}

\img<exmp>[0.65]
{Example execution of the code for the first three questions.}
{media/example.png}

\section{Submission Instructions}

Please \textit{replace the professor's last name} with yours wherever appropriate.

\begin{itemize}
  \item Submit \texttt{OneTashfeen.java},
        \texttt{TwoTashfeen.java}, \texttt{ThreeTashfeen.java},
        \texttt{FourTashfeen.java}, \texttt{FiveTashfeen.java} and
        \texttt{solTashfeen.pdf} at the online classroom.

  \item The files \texttt{OneTashfeen.java},
        \texttt{TwoTashfeen.java}, \texttt{ThreeTashfeen.java},
        \texttt{FourTashfeen.java} and \texttt{FiveTashfeen.java} should
        contain the Java source code for the relevant questions. E. g.,
        \texttt{OneTashfeen.java} should contain the code that pertains to
        the first question, \texttt{TwoTashfeen.java} to the second and so
        forth.

  \item The PDF file \texttt{solTashfeen.pdf} should contain
        written answers to questions that ask for them as well as a
        screenshot similar to the one in figure \ref{exmp} that
        demonstrates your code being compiled and ran.
\end{itemize}

\end{document}
